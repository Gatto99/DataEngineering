\documentclass{article}
\usepackage[margin=50]{geometry}
\begin{document}
\title{Homework 1}
\author{Davide Gattini}
\date{14/10/2022}
\maketitle

Andrew NG, globalmente riconosciuto nel mondo dell'intelligenza artificiale, ha identificato un grande cambiamento in questo ambito: sostiene che l'adozione dell'approccio \textit{data-centric}, per cui è sufficiente avere a disposizione pochi dati ma di qualità, sia una soluzione migliore rispetto ad avere un dataset con molti dati, usati per addestrare un modello che successivamente si pensa a modificare per ottenere risultati migliori (approccio \textit{model-centric}). Quindi, secondo Andrew NG, collezionare sempre più dati risulta costoso e meno efficace che occuparsi dell'ingegnerizzazione dei dati (che quindi prevede la pulizia di quest'ultimi) e, poiché è un approccio già adottato, per evitare che rimanga solo un'intuizione, crede necessario che esso diventi sistematico.


L'utilizzo di quei modelli che Andrew NG chiama \textit{foundation models}, potrebbe risultare vantaggioso qualora si disponesse di una quantità di dati sufficiente per addestrare il modello e renderlo utile ed efficace per differenti tasks. Però non sempre questo requisito viene soddisfatto, infatti come Andrew NG sottolinea, ci sono realtà in cui il dataset a disposizione è "povero", quindi è conveniente e produttivo seguire l'approccio data-centric. Di contro, è necessario trovare del personale qualificato, che lo stesso Andrew NG chiama \textit{MLOps}, che deve curarsi di questo aspetto per bilanciare una bassa quantità di dati con una buona qualità degli stessi, ma molto spesso tutto ciò viene già fatto, in quanto è sempre prevista l'acquisizione dei dati e la loro organizzazione, in modo che siano utilizzabili e coerenti.

E' dunque furbo ed efficace risolvere i problemi specifici cercando di dare in pasto al modello dati specifici con cui colmare l'inconsistenza dovuta ad un addestramento errato. Quindi continuare a raccogliere un quantità indefinita di dati (a volte non disponibili) potrebbe risultare come una perdita di tempo e risorse che potremmo impiegare a migliorare dati che già abbiamo e conosciamo. Se ci si dovesse domandare cosa fosse meglio, se migliorare il modello o i dati, penserei a migliorare i dati: a parità di dati, migliorare il modello potrebbe sì risolvere il problema in maniera più o meno efficace, ma il modello adottato potrebbe comunque comportarsi diversamente da quello che ci aspettiamo, poiché, a causa del rumore con cui mal addestriamo l'architettura (qualsiasi essa sia), non sarà efficace quanto lo \textbf{stesso modello} che alimentiamo con gli \textbf{stessi dati} ma privati del rumore.

\end{document}